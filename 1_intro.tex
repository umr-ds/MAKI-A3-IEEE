\section{Introduction}
\label{sec:intro}

Modern networked systems are based on hardware and software components that have a communicative relationship with each other to enable applications, be it a network interface, or a system call to write data to disk.
However, many of these components are outdated or lack functionality that do not optimally enable communication between the components for a given situation.
If these components are proprietary and the manufacturer is not willing or able to add a certain function or update a component, users often have no choice but to accept the non-optimal functioning of the networked system.
There can be various, even legitimate, reasons why manufacturers do not update a component.
Besides simple business reasons, such as a lack of or insufficient monetization of deprecated components, technical hurdles such as missing infrastructure can also be a problem.
Long and costly standardization and deployment processes can also hinder or slow down the updating of components, which is not only a problem of proprietary components.

To overcome this problem, developers, companies and organizations already use abstractions and work-arounds to add or modify functionality to/of existing networked systems or applications without touching the proprietary components by adding mechanisms that intercept and modify the communications between the component and its environment.
As an example, due to the relative low number of available IPv4 addresses network developers introduced Network Address Translation (NAT), where a gateway intercepts IP packets and modifies translates the IP addresses between private and public addresses.
Another example is Wine, that allows executing Windows applications on Unix-like operating systems by intercepting the applications Windows system calls and translating them to the corresponding POSIX system calls.

We argue that such approaches are not exceptions or simple work-arounds but can be found throughout modern networked systems.
We even argue that such approaches are essential to modern networked systems to provide fast, widely used improvements of existing systems.
However, until now, there is no model that can be used by developers to systematically identify possible starting points and implement such intercepting layers between two communicating components.
Therefore, we present \emph{\mm}, a novel approach providing a model to implement unobtrusive functional additions to or modifications of an existing networked system without touching any proprietary components.
By classifying a networked system into the \emph{system}, i.e., components containing proprietary or other not changeable parts, and the \emph{environment}, i.e., components containing modifiable parts under control, 
behavioral changes are achieved by mechanism-enhancing yet unobtrusive \emph{interceptors},
i.e., a layer that is introduced between environment and system adding or updating mechanisms.
Such an interceptor can have many forms like an updated software library, a newly deployed edge device, or an enhanced cloud service.
Interceptors must be unobtrusive to avoid disrupting or even breaking applications but still provide added or modified mechanisms to the networked systems.

We illustrate this approach by applying the findings of the model in two case studies
that show its real-world applicability.
First, we unobtrusively replace a cloud infrastructure by an edge infrastructure in a wireless sensor network.
Second, we unobtrusively add a vertical handover mechanism between Wi-Fi and LTE to a mobile end device without disconnecting TCP sessions.
Our results indicate that \mm is a compelling approach to achieve improved service quality and provide previously unavailable functionality in an unobtrusive manner.

The contributions of this paper are:
\begin{itemize}
    \item We present \mm, a model that can be used by developers to design and implement unobtrusive mechanism interceptors.
    \item We present two examples, NAT and Wine, which serve as a basis for developing the \mm model.
    \item We present an application, \ttt, where our model is applied to replace a cloud infrastructure by an edge infrastructure to unobtrusively add previously unavailable functionality in a sensor network.
    \item We present an application, where our model is applied to add an unobtrusive vertical handover mechanism between Wi-Fi and LTE and thereby improve service quality of mobile devices.
\end{itemize}

The remainder of this paper is organized as follows.
Section~\ref{sec:releted_work} presents related work.
Section~\ref{sec:design} introduces two real-world examples and the derived model for \mm.
Section~\ref{sec:treetalker} presents the first case-study, \ttt and Section~\ref{sec:wg:impl} presents the second case-study, \wgh.
Section~\ref{sec:conclusion} concludes this paper and presents future work.
