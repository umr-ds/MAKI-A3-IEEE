\section{Introduction}
\label{sec:intro}

In our daily life, most devices that we use require communication. 
This can be a smartphone, a smart light bulb, the car radio, or many others.
Nearly all these devices have a communications interface, be it a network interface, or a system call to write data to disc. 
A system without communication is nearly useless in most cases. 
However, communication systems are not only increasingly used in private life, but also in all conceivable professional areas; for example in industry, computer-supported applications have become indispensable.  
This increased and long term use of interconnected systems raises the question of how they can be supplied with updates over time. 
With open source applications and open source operating systems this may work for experienced users, but for proprietary applications, components or systems this is only possible through the manufacturers.
Especially in the industry, where special-purpose systems are the norm, many devices exist that have this dependence on manufacturers. 
But there are also many other areas were proprietary software, components, and systems dominate the market.
As long as the vendor is willing and able to provide the systems with updates, the hardware and software components may continue to be supplied with innovations, but if not, then the systems are frozen at their level of functionality. 
Thus, the scope of functionality depends solely on the will of the manufacturers.
Also, new challenges, such as security and the introduction of new technologies, often lead to new mechanism not being transferred to all hardware and software components. 
This leads to these systems becoming legacy more quickly than would be necessary. 
By a mechanism, we refer to the concrete implementation of a functional unit that is used by the system to achieve its task. 

In particular, network systems and applications employ many non-proprietary protocols that should be more interchangeable and extensible by design. 
But the protocols used in networked systems today have been standardized over a long period of time and implemented in operating systems and devices by vendors.
Much of the internet traffic is based on TCP  \cite{A3:john2007analysis}, which is standardized by the IETF and requires a lengthy process to integrate new mechanisms\cite{A3:de2019pluginizing}. 
For example, an approach has been presented in the literature that adds network coding to TCP \cite{A3:sundararajan2011network}. 
These new mechanisms are only useful if they are widely adopted and both sender and receiver have this extension. 
Another way would be to publish a new protocol with the new mechanisms instead of extend a existing protocol, but the problem remains the same. 
This creates a fundamental problem: vendors who switch to a new standard or extensions early on have hardly any competitive advantages, but without advantages of their own, hardly any vendor will switch.
This means that it takes longer to achieve substantial market penetration, as a certain critical mass of users must be exceeded. 

But communication protocols are only one level at which this problem occurs, as well as special parts of communication protocols like congestion control or even whole network concepts like DTN or even transmission technologies. 
All these categories have the problem that first enough systems have to be accessible for the new mechanism to take advantage of the usage. 
This is also true for security mechanisms that should be used for security reasons but are not available in legacy or unmodifiable systems. 
Here, the \mm can contribute to the security of systems by making these security mechanisms available.

In order to achieve the transition capability of networked systems and proprietary hardware/software components in practice, bringing new mechanisms to this systems must be achieved in as many cases as possible. 
To maximize the benefit for early adopters a method to get new mechanisms to legacy devices called  \mm seems to be promising.
On the one hand a \mm on the operating-system-level could be a good way to break the innovation blocking point of gaining nearly no benefits for the vendor/user, by intercepting communication and integrating new mechanisms to legacy apps and supporting new mechanisms os wide.
In this case the apps are our target system for our \mm and the os is our environment where we can make chances and unobtrusive intercep the app communication.
On the other hand the network itself could wrap non-changeable legacy devices, our target system, to virtually bring new mechanisms to this devices to support a wide range of them and enable an easy way of bringing new technology much faster to legacy devices and spread new technology. 
In this example our environment for \mm clearly is the network itself and with this environment we bring new mechanisms to the target system by \mm. 


% 
% 


Our contributions are:

\begin{itemize}
 \item A concept to break the innovation blocking point of gaining nearly no benefits for the vendor/user, by intercepting communication and integrating new mechanisms.
 \item A concept to bring new mechanisms to proprietary hardware/software components
 \item Two case studies to show the feasibility of our concept 
\end{itemize}

The paper is organized as follows. 
Section~\ref{sec:releted_work} discusses related work.
In Section~\ref{sec:design}, the concept and design of unobtrusive \mm are presented.
Section \ref{sec:treetalker} describes the first case study with out \ttt implementation and evaluation. 
In \ref{sec:wg:impl} our second case study for unobtrusively inserting vertical handovers is presented. 
Section~\ref{sec:conclusion} concludes the paper and outlines areas of future work.