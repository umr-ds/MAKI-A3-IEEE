\begin{abstract}
Networked systems and applications are often based on proprietary hardware/software components that manufacturers might not be willing to adapt or update if new requirements arise.
%Furthermore, to deploy new hardware/software components, long development and standardization processes are typically required.
We present \emph{mechanism interception}, a novel approach to unobtrusively add or modify functionality to/of an existing networked system or application without touching any proprietary components. Behavioral changes
are achieved by functionality-enhancing yet unobtrusive interceptors, e.g., an updated software library, a newly deployed edge device, or an enhanced cloud service.
We illustrate our approach by two case studies
that 
%highlight its advantages and 
show its real-world applicability:
(a) we unobtrusively replace a cloud infrastructure by an edge infrastructure in a wireless sensor network, and
(b) we unobtrusively add a vertical handover mechanism between WiFi and LTE to a mobile end device without disconnecting TCP sessions.
Our results indicate that mechanism interception is a compelling approach to achieve improved service quality and provide previously unavailable functionality in an unobtrusive manner.
\end{abstract}


\begin{IEEEkeywords}
    Network Protocols, Embedded Devices, Sensor Networks, Edge Computing, Mechanism Interception
\end{IEEEkeywords}