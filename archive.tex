Figure~\ref{fig:bigpicture} shows the interaction of \emph{mechanism migration} with \textit{deformable transitions}.



The non-transitionable communication system uses mechanism $A$ to fulfil its functionality.
The mechanisms $B'$ and $C'$ are migrated into the non-transition-capable communication system, i.e. the mechanism $B'$ is made usable for the system e.g. by pre-positioning (e.g. by a proxy, illustrated by the filled grey diamond), and  
the mechanism $C'$ is made accessible to the system e.g. by modifying the system (e.g. by changing a library used, illustrated by the filled grey circle).
A deformable transition can be used under different conditions, e.g. a reduced choice of mechanisms or a limited basis for decision-making (e.g. missing information from the non-transitionable communication system; indicated by the X in the arrow of the data flow to the deformable transition); the transition logic remains intact. 
The transition can be automatically deformed by pre-processing alternative information sources (indicated by the operators to the left and right of the transition logic) to be usable under given conditions. 

Mechanism migration is not limited to the area of communication, but can also be used in the other areas of computation and data storage.
A photo application can serve as an example of mechanism migration in computation.
The application stores captured images in a predefined format that can be replaced by another format by means of mechanism migration in order to enable more efficient storage utilisation, for example through new compression techniques. 
Conceivable extensions for this example are the outsourcing of calculations to special hardware (integrated FPGAs) or edge services.
An illustrative example of the migration of a memory mechanism is a hybrid semi-volatile memory that accelerates read access times by orders of magnitude and thus improves the quality of service of a programme. 
In addition, intelligent caching mechanisms can be migrated in combination with network resources so that a high quality of service is achieved even with poor network access. 

In order to use cooperative transitions together with a non-transitionable communication system, this communication system must either be modified (e.g. by implanting code) or indirectly equipped with new functionalities by pre-positioning additional functional units. 
By using deformable transitions with the help of data-driven approaches in communication systems with other decision bases and mechanisms, the automated construction of transition logics is supported. 
The transition logic is viewed as a self-contained module that has fixed interfaces for inputting decision bases and outputting transition decisions.
The interfaces can be assigned various information sources in a deformable transition, which are adapted in preprocessing by means of simple logical and arithmetic operations, as well as by means of more complex data-driven algorithms.
