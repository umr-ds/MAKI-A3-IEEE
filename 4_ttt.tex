\section{\ttt}
\label{sec:treetalker}

To demonstrate the power of the network-side adaption, we decided to adapt a legacy-device which has been in real-world use for some time.
We decided on the ''Treetalker'' platform\footnote{\url{https://www.nature4shop.com/}}, which is a product for distributed tree-health monitoring.
This platform has been in use in many areas in the world for example in Moscow city with 250 TreeTalkers or a cooperation with the Peking University. \footnote{https://www.nature4shop.com/our-vision/}

The platform vendor sells both the sensor-devices (henceforth simply called \textit{Treetalker}), as well as a basestation (henceforth called \textit{TTCloud}), which can be used to aggregate data via LoRa and upload it to a proprietary cloud-storage provider via GSM.
The ''Treetalker''-platform is proprietary, and does not give buyers access to the device firmware.

For our project, we aimed to replace the static \textit{TTCloud}, with a more versatile gateway, to allow for dynamic reconfiguration of all \textit{Treetalkers}.
We call the resulting system \ttt.

The TTCloud, and its associated Treetalkers, can be configured via text-message, also, collected data is sent to the cloud-backend via a mobile data connection\footnote{the operator needs to provide their own SIM-card with a Text-/Data-Plan}

While LoRa-alliance certified hardware provides a dedicated network-layer-protocol, called \textit{LoRa-WAN} which deals with collisions, addressing, and other issues resulting from the shared-medium characteristic of LoRa, the \textit{Treetalker}-vendor decided to forgo this higher-level protocol in favor of using the LoRa-PHY-layer directly.

For this reason, communication with each \textit{Treetalker} needs to be scheduled manually in such a way that avoids collisions.

In order to replace the \textit{TTCloud} with our own gateway, we performed a blackbox-analysis of \textit{Treetalker's} behaviour and network communications, to understand the communication protocol.

Communication between the TTCloud and its sensor-stations is a call-response-protocol, initiated by the TreeTalker.
The TTCloud's command message contains 4 fields, with which the Treetalker's behaviour can be influenced.
\textit{Sleep} is the time that the device will sit idle in between measurements (defaults to 1 hour).
\textit{Heating Time}, that is how long the Treetalker will run its internal heater before taking the second set of heat-probe measurments.
Lastly, we have the \textit{time slot} and \textit{slot length}.
These values govern the time that the device will wait between measurements and sending measurement data, it appears that these two values are simply multiplied to get the time-to-wait before transmitting.

In normal operation, with a vendor-supplied gateway, the sleep- and heat-time are user-configurable, but fixed, and the same for all Treetalkers attached to the gateway.
The time slot length will also be the same for all nodes, while the gateway will assign each Treetalker a unique time slot.
Again, this is necessary due to the vendor's refusal to use the LoRaWAN-protocol, which includes a mechanism for collision-avoidance, but since the Treetalker-system uses the raw \textit{LoRaPHY}-layer, the gateway needs to schedule transmission manually.
For this reason, there is also no way for a Treetalker-network to coexist with other LoRa-networks in the same area, since collisions would be plenty and unrecoverable.

With the insights gathered by our analysis we were able to fully replace and improve the gateway functionality, which allowed us to bring \mm to the treetalker-network.

\subsection{System Design}
\label{sec:treetalker:design}

Based on the ring-model described in section \ref{\label{sec:design}}, this scenario can be expressed as a 3-ring design.
The innermost ring, the Treetalker's firmware is unaccessible to us
