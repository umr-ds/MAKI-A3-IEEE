\section{\ttt}
\label{sec:treetalker}

% For the first case study, we decided to adapt a legacy-device which has been in real-world use for some time.
The \emph{TreeTalker} platform is a product for distributed tree-health monitoring.
This platform has been in use in many areas in the world for example in Moscow city with 250 TreeTalkers or a cooperation with the Peking University\footnote{\url{https://www.nature4shop.com/our-vision/}}.
The University of Marburg also currently operates 100 TreeTalkers as part of the \textit{Nature 4.0} research project\footnote{\url{https://www.uni-marburg.de/en/fb19/natur40}}.

The platform cosists of sensor-devices (henceforth simply called \textit{TreeTalker}), as well as a base station (henceforth called \textit{TTCloud}), which can be used to relay data via LoRa and upload it to a proprietary cloud-storage provider via GSM.
The TreeTalker-platform is proprietary, and does not give buyers access to any hardware or software components.
For our project, we aimed to replace the static \textit{TTCloud}, with a more versatile gateway, to allow for dynamic reconfiguration of all \textit{TreeTalkers}.
We call the resulting system \ttt.

The TTCloud, and its associated TreeTalkers, can be configured via text-messages and collected data is sent to the cloud-backend via a mobile data connection using a proprietary call-response protocol, initiated by the TreeTalker.
% \footnote{the operator needs to provide their own SIM-card with a Text-/Data-Plan}
While LoRa-certified hardware provides a dedicated network-layer-protocol, called \textit{LoRa-WAN}, which deals with collisions, addressing, and other issues resulting from the shared-medium characteristic of LoRa, the \textit{TreeTalker}-vendor decided to forgo this higher-level protocol in favor of using the LoRa-PHY-layer directly.
For this reason, communication with each \textit{TreeTalker} needs to be scheduled manually in such a way that avoids collisions.
This means that it is not possible to operate a Treetalker network at the same location as a LoRa-WAN network, since the transmissions from these networks will collide.

% In order to replace the \textit{TTCloud} with our own gateway, we performed a blackbox-analysis of \textit{TreeTalker's} behavior and network communications, to understand the communication protocol.

The TTCloud's command message contains 4 types for influencing the TreeTalker's behavior.
\textit{Sleep} is the time that the device will sit idle in between measurements (defaults to 1 hour).
\textit{Heating Time}, that is how long the TreeTalker will run its internal heater before taking the second set of heat-probe measurements.
Lastly, we have the \textit{time slot} and \textit{slot length}.
These values govern the time that the device will wait between measurements and sending measurement data, it appears that these two values are simply multiplied to get the time-to-wait before transmitting.

In normal operation with the TTcloud, the sleep- and heat-time are user-configurable during deployment, and the same for all TreeTalkers attached to the gateway.
The time slot length will also be the same for all nodes, while the gateway will assign each TreeTalker a unique time slot as a result of using LoRa and not LoRa-WAN.
% Again, this is necessary due to the vendor's refusal to use the LoRaWAN-protocol, which includes a mechanism for collision-avoidance, but since the TreeTalker-system uses the raw \textit{LoRaPHY}-layer, the gateway needs to schedule transmission manually.
% For this reason, there is also no way for a TreeTalker-network to coexist with other LoRa-networks in the same area, since collisions would be plenty and unrecoverable.
Finally, the system has no interactivity.
The only way for a user to access the collected data is by downloading a CSV-file containing the measurement data from the vendor's web server.

With the insights gathered by our analysis we were able to fully replace and improve the gateway functionality, which allowed us to bring \mm to the TreeTalker-network.

By intercepting and redirecting Treetalker messages, we can increase both the functionality, and reliability of the Treetalker-network.
We can enable interactivity, where previously the system was static, we can perform statistical analysis on measurement data to find and rectify anomalies by triggering remeasurements, and lastly, we can maximise battery lifetime by adjusting measurement frequency to the remaining battery power.

\subsection{General Design}
\label{sec:TreeTalker:design}
% System: Alle TreeTalker, Ziel: Daten generieen und sie verfügbar machen
% Environment: LoRa Gateway (normalerweise TTCloud) -> Internet
% Mechanism: Lora 
% Interceptor: TTT + Remote (zusammen ForestEdge)
% Unobtrusive: TreeTalker soll keinen Unterschied zw. TTC und ForestEdge erkennen können

\begin{figure}
    \centering
    \includegraphics[width=\linewidth]{figures/model_TTT.pdf}
    \caption{The system architecture for the TreeTalker scenario with interceptor, system and environment as introduced by figure~\ref{fig:model}}
    \label{fig:model_TTT}
\end{figure}

The system in this case study is a swarm of deployed TreeTalkers in a forest,
surrounded by the forest-wide LoRa network serving as a gateway to connect the TreeTalkers to the Internet as the environment.
The mechanism used by the TreeTalkers is LoRa.
Finally, the proposed \ttt forms the interceptor.
% , its goal is the collection of accurate environmental data and to make it available to users.
% b) As for the \textit{environment}, it takes the shape of a forest-wide LoRa network which serves as gateway to connect the TreeTalker swarm to the internet.
% It receives the TreeTalker's data via LoRa and can both process it and relay it onwards to other systems on the internet via a variety of mechanisms, such as WiFi, LTE, 5G or satellite uplink.
% c) The \textit{mechanism}, which allows the TreeTalker swarm to talk to the system, is LoRa.
% d) \textit{\ttt} is the \textit{interceptor}.
It replaces the TTCloud, communicates with the TreeTalker swarm via LoRa, and ultimately allows users to access the data in a more convenient format.
Furthermore, it enhances the functionality of the system as a whole by introducing anomaly detection and therefore improve the quality of the collected data.
This approach is unobtrusive, since the TreeTalkers are unable to distinguish between the original TTCloud and \ttt.


Fig. \ref{fig:model_TTT} shows the architecture of the \ttt system.
Our architecture employs two separate decision making instances, one is located on the physical device in the forest, which communicates directly with the TreeTalker swarm, and only has data on its directly connected sensors-nodes.
This component is labeled \ttt in the figure; it consists of the actual interceptor, shown in yellow, which includes the LoRa transceiver adn a translation-layer which remarshals the packets and sends them on via MQTT.
The anomaly-detection checks if a measurement seems reasonalbe, and if it passes, the packet is transmitted onward via a number of different communication technologies.

The second, centralized aggregator instance on the backend, has global knowledge of all nodes in the network.
It receives all the data from all \ttt instances and saves them in a time-series database for further evaluation.
It also contains a presentation-layer through which users can view the data.

When the local decision engine receives a packet, it both stores the data locally and also forwards it to the global aggregator, which periodically aggregates all the data from all the nodes and computes relevant benchmark values. 
These are then distributed to all interceptors.
When generating a reply packet, the local decision engine performs a statistical evaluation of the respective treetalker's previous measurement data, and how it relates to the aggregator-supplied benchmark values.
If it detects an anomaly, it may require the TreeTalker to rerun its measurements by sending a specially crafted command packet with a short sleep time.

In order to make meaningful decisions, concept recognition is necessary.
This is done by comparing a long-running, and short-running time-window of the data.an
It has been shown that the calculation of the twofold standard deviation for short-running windows is already sufficient to be able to make decisions - for example, whether a new measurement is necessary or these new values can be appointed as the new standard.
In general, the preferred decision is to reduce the upcoming measurement interval in order to verify and increase the accuracy of the collected data.
Due to the minor costs on battery and data transmission over Lora, even a 20\% increase in measurements is insignificant for the overall energy consumption.

If no anomaly is found, the sleep time until the next measurment is calculated based on the Treetalker's reported remaining battery capacity.
This serves to maximise the device's lifetime under battery by reducing the measurement frequency during times when the pohtovoltaic-panel produces little energy, and increasing the frequency when there is a power surplus to increase resolution.

\subsection{Evaluation}
\label{sec:TreeTalker:evaluation}

\subsubsection{Response Time}
\label{sec:evaluation:response-time}

The protocol employed by the TreeTalker is a simple two-way exchange of data, where the TreeTalker will send measurements to the associated TTCloud and wait for new configuration parameters.
If it does not receive a response to its data packet after a certain time, it assumes that the cloud is offline and falls back to repeating the message ad-infinitum.
Thus, in order to be a viable alternative to the first-party TTCloud, the \ttt needs to be able to respond to each received data-packet within reasonable time.

In order to compare the response time of our system versus the TTCloud, we used a experiment setup of a single TreeTalker, in combination with a passively sniffing node to recorded the LoRa-traffic for further analysis.

\begin{figure}
    \centering
    \includegraphics[width=\linewidth]{figures/response_times.pdf}
    \caption{Response time of both the first-party \textit{TTCloud}, as well as \textit{\ttt}}
    \label{fig:response_times}
\end{figure}

As can be seen in figure \ref{fig:response_times}, the \ttt is capable delivering responses with a similar delay as the TTCloud.
The three bars are, from left to right: 1) The total response time of \ttt, including both the time it takes to generate a reply packet, as well as transmitting it via LoRa. 2) The response time of the first-party TTCloud, also including LoRa Transmission. 3) The time it takes \ttt just to generate the reply packet.
In fact, the actual time it takes to arrive at a decision and craft the reply-package is less than one quarter of the total time, the other three quarters being the time necessary to transmit the packet via LoRa.
The TTCloud, which does no further processing whatsoever, does have the advantage of only sending the same reply every time, thus not requireing any additional time for computations of any kind.
Therefore, it is sometime very slightly faster, though not by any significant margin.

While there are occasional outliers, wherein the analysis of a packet and the generation of the appropriate response packet take significantly longer, these are very rare, being 4 out of 200 cases.
This is most likely due to database-queries occasionally taking longer, probably due to scheduling issues.

\subsubsection{Anomaly Detection}
\label{sec:evaluation:anomaly-detection}

To evaluate whether the anomaly-detection functions as intended, we drew upon real-world data gathered by a TreeTalker-deployment, which was created as part of the \textit{Nature 4.0} project of the University of Marburg.
This deployment is comprised of 100 TreeTalker which were set up in a university-owned parcel of forest and has been continually gathering sensor data for the past 2 years.

The evaluation was performed by injecting historical data into the policy-engine in chronological order, thus simulating a replay of the 2 year data-gathering-process.

\begin{table}[tb]
    \centering
    \caption{Anomalous and critical events found in historical data}
    \label{tab:evaluation:anomalies}
    \begin{tabular}{lcccc}
        \toprule
        Type & Position & Movement & Stem Temp. & Air Temp.  \\ \midrule
        Anomalous & 2182 & 108 & 443 & 0 \\
        Critical & 6313 & 105 & 183 & 1 \\
        \bottomrule
    \end{tabular}
\end{table}

As becomes obvious from table \ref{tab:evaluation:anomalies}, the vast majority of events are produced by the accelerometer, and more specifically by the values denoting the sensor's position.

Further investigation of the data revealed that this seems to be due to a systematical fault in either the installed sensor itself, or the method in which this sensor is queried by the onboard electronics.
The data reported by the TreeTalker for this sensor seems to be completely erratic and entirely unsuited as a basis for further decision-making.
This fault had previously not been noticed by the operator and could have wrought havoc on data analysis performed on the gathered data, but with the aid of \ttt, we were able discover this deficiency before it could impact further decision-making.

\begin{figure}
    \centering
    \includegraphics[width=\linewidth]{figures/TTT_position_data_smallest.png}
    \caption{Erratic data from the deployed TreeTalker in the Marburg Open Forest}
    \label{fig:ttt_position_error}
\end{figure}

In conclusion, by applying our interception-concept to the TreeTalker-network, we were able to significantly improve the system's functionality and performance.
We were also able to leverage behavioral modification to enable interactivity on an otherwise static system, which allowed us to recognize faulty sensors and prevent corrupted data from derailing other research projects.
