\section{Related Work}
\label{sec:releted_work}
The challenge of applying novel protocols and mechanisms to existing systems is discussed in different ways in the literature. 
In particular, the two areas of inherent extensibility of network protocols, and userspace network protocol implementations are relevant in this area and hence are discussed in this chapter. 

% \paragraph{Inherent Extensibility of Network Protocols}
Foundations for making network protocols extensible and updatable lie in the active networks research branch \cite{A3:tennenhouse1996towards, A3:tennenhouse1997survey}.
ANTS \cite{A3:wetherall1998ants} provides a system architecture that enables new protocols to be deployed on routers as well as terminals through platform-independent code.
Furthermore, a mechanism was presented in the literature that allows new transport protocols to be deployed easily and quickly \cite{A3:patel2003upgrading}. 
The protocols are exchanged between two communication partners, selected at the beginning of a connection and executed in the kernel.
For example, Pathak et al. \cite{A3:pathak2015modnet} present a system that provides applications with additional TCP interfaces and makes TCP adaptable in a fine-grained way, but has hardly been used by applications so far.
With software-defined networks \cite{A3:mckeown2008openflow} and the network programming language P4 \cite{A3:bosshart2014p4}, network programmability was achieved. 
Tran et al.~\cite{A3:tran2019beyond} present a method to add functionality to TCP kernel implementation using eBPF. 
As examples, the subsequent implementation of the user timeout option, as well as the dynamic change of the TCP congestion control mechanism are investigated.
QUIC is an alternative to TCP brought forth first by Google and then by the IETF, but is intended to be advanced in a similar manner to TCP.
De Coninck et al.~\cite{A3:de2019pluginizing,A3:de2018tuning} present a method to dynamically tune QUIC on a per-connection basis through extensions. 
The presented approaches can be used by actively adapting existing systems and then enable the extensibility of these systems.

% In summary, the current state of research still lacks strategies that provide for the inherent extensibility of new protocols and mechanisms -- even for domains where continuous evolution has been shown to occur in recent years. 
% In addition, proprietary system components make it difficult to easily exchange mechanisms on common end devices. 


% \paragraph{Userspace Network Protocol Implementations}
Network stacks are usually implemented in the operating system kernel for efficiency reasons, and existing applications use this kernel implementation. 
However, network protocols can also be implemented independently of the kernel to allow faster development rates and wider distribution. 
Alpine introduces an alternative network stack in which changes can be implemented and tested quickly \cite{A3:ely2001alpine}.
MultiStack \cite{A3:honda2014rekindling} is a userspace implementation of a network stack framework that can provide isolated network stacks for different applications and enables rapid extensibility. 
An implementation of TCP in userspace is presented by Jeong et al \cite{A3:jeong2014mtcp}.
By executing in multiple threads, it is possible to achieve orders of magnitude higher bandwidth on multi-core machines.
With NUSE, the network stack of the Linux kernel can be used and further developed as a userspace library \cite{A3:tazaki2015library}. 
Existing programs can thus use modified protocols without modifications to the program itself.
Heuschkel et al.~\cite{A3:heuschkel2016virtualstack} present VirtualStack, which allows different userspace network stacks to be used on one system. 
In these network stacks, extensions as well as new protocols can be implemented and used quickly, and applications can benefit from the protocol changes in the virtual stack without adjustments. 
ClickNF \cite{A3:gallo2018clicknf} is an extension of a software-defined router in which the lower four layers of the network stack can be exchanged in a modular way. 
In order to adapt network protocols and the network stack, a system has been presented in the literature with the help of which the network stack can be detached from virtual machines (VM) and centrally managed on the host operating system and made available to the various VMs \cite{A3:niu2019netkernel}. 
SocksDirect \cite{A3:li2019socksdirect} offers the possibility for efficient socket-based communication via local networks. 

% In summary, it can be stated that despite a high number of related approaches for processing network protocols in user space, the corresponding solutions have not made broad inroads into existing communication systems. 

% NAT als related Work / als Interceptor