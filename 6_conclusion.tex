\section{Conclusion}
\label{sec:conclusion}

We presented \mm, a novel approach providing a model to implement unobtrusive functional additions to or modifications of an existing networked system without touching any proprietary components.
We classifying a networked system into the system and the environment so that
behavioral changes are achieved by mechanism-enhancing yet unobtrusive interceptors,
We illustrated this approach by applying the findings of the model in two case studies that
show its real-world applicability:
(a) we unobtrusively replaced a cloud infrastructure by an edge infrastructure in a wireless sensor network, and
(b) we unobtrusively added a vertical handover mechanism between Wi-Fi and LTE to a mobile end device without disconnecting TCP sessions.
The results of the case studies indicate that \mm is a compelling approach to achieve improved service quality and provide previously unavailable functionality in an unobtrusive manner.

There are several areas of future work.
Currently, to apply the presented approach, every step, starting from the identification of the system, environment, mechanism and possible interceptors, requires manual work per application.
Providing tools, automations and infrastructure to reduce manual work would help to booster the adoption of our model.
Besides adding a single mechanism as part of the interceptor to enhance a single given proprietary component, future iterations should consider supporting multiple mechanisms per system and multiple systems per interceptor.
Finally, when an interceptor replaces or modifies an existing mechanism or adds a new mechanism, the old mechanism is still available, although not used.
Transitioning between multiple mechanisms could add the benefit to use the mechanism that achieves the best results for the given situation.